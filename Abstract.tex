

%abstract---------------
随着城市交通电气化进程的快速推进,城市轨道交通的能耗快速增长,城轨系统需求侧响应成为了一种有效的降低牵引能耗成本的手段。本文针对城轨系统的能耗问题,探讨了不同运行策略对能源消耗的影响,通过分析列车运行过程中的路段坡度、速度限速等因素,建立了列车消耗能量优化模型,并结合栅格化蚁群算法进行优化求解。

针对问题一,我们将列车运行阶段进行划分并分别建立距离,速度与时间的微分方程,通过4阶龙格库塔法求其数值解。考虑列车在最短时间上增加运行时间的运行速度轨迹,并画出分别增加10s、20s、50s、150s、300s的速度-距离曲线、牵引制动力-距离曲线、时间-距离曲线与能量消耗-距离曲线。

针对问题二,我们将路段进行划分并栅格化,利用栅格化蚁群算法对其进行求解,得到使运行过程能耗降低的可行的速度轨迹。在考虑路况信息情况下利用问题一模型求解得到列车最短运行时间的速度轨迹。根据问题一,画出列车以最短时间到达站台 B、在最短运行时间上分别增加 10s、20s、50s、150s、300s 到达站台 B的6组曲线。

针对问题三,我们将列车匀速运行分为两个阶段来降低列车在匀速阶段行驶距离,使得列车能够在原计划基础上延迟60s到达终点,并且对运行方案进行优化,得到优化后的运行速度轨迹并做出优化后列车运行过程的速度-距离曲线、牵引制动力-距离曲线、时间-距离曲线与能量消耗-距离曲线。





